% !TEX root = main.tex
\section{Results} % (fold)
\label{sec:results}

% Get stats (bond distances and angles) from the 4 simulations.

In order to evaluate the simulations, we made several measurements for each simulation and compared these measurements across the different versions.
\figref{unphosphorylated_bond3} shows the distance between termini for the unphosphorylated fragments starting with extended and native conformations.
The unphosphorylated extended structure folds during the simulation but does not conform to a stable state,
the distance between its termini varying significantly over time.
In contrast, \figref{phosphorylated_bond3} shows that neither the extended nor native peptide changed conformation signifantly;
the distance between the termini remained constant throughout both simulations.
This could be due to the stabilizing effect of phosphorylation or simply because the simulation was not run long enough.
The simulations that produced these figures used a 10{\AA} solvation box.

\figref{unphosphorylated_extended_states} shows the conformations of unphosphorylated extended structure at different stages of simulation.
The structure samples a variety of conformations ranging from largely extended to compact, with T37 and T41 briefly coming close and then separating again.

To determine whether or not the phosphorylation of T37 affected the secondary structure of the fragment,
we calculated the $\phi/\psi$ angles of multiple residues in the peptide, as shown in \figref{ramachandran_native_s8},
which is comprised of all $\phi/\psi$ in all simulations, divided into phosphorylated and unphosphorylated simulations.
The phosphorylated peptide sampled a narrower range of $\phi/\psi$ angles than its unphosphorylated counterpart,
suggesting that the structure imposed by the phosphorylation restricts the favorable conformations of the fragment.

% Data for this figure came from:
% 2mx4_up1_extended_s10_run1
% 2mx4_up1_s10_run1_spm8
\addwidefigure{unphosphorylated_bond3}{1}{Distance between termini for unphosphorylated fragments starting with extended and native conformations.}

% Data for this figure came from:
% 2mx4_p1_extended_s10_run1_c1
% 2mx4_p1_s10_run1_c1
\addwidefigure{phosphorylated_bond3}{1}{Distance between termini for phosphorylated fragments starting with extended and native conformations.}

% Data for this figure came from:
% all simulations
\addwidefigure{ramachandran_native_s8}{1}{Each dot is a $\phi/\psi$ angle of residue taken at a different time point in some simulation.}

% Data for this figure came from:
% 2mx4_up1_extended_s10_run1
\addwidefigure{unphosphorylated_extended_states}{1}{Conformations of unphosphorylated extended structure at different stages of simulation.}

% Figures: bond distances vs time for all simulations

% Figures: angles vs time for all simulations

% section results (end)
