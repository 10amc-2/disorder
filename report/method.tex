% !TEX root = main.tex
\section{Methods} % (fold)
\label{sec:methods}

\subsection{Producing the fragment PDB file}
The NMR structure of 4ebp2 used in this project is the same as in Bah \textit{et al.}
and has PDB ID \href{http://www.rcsb.org/pdb/explore/explore.do?structureId=2MX4}{\textcolor{blue}{2MX4}}.
Because 2MX4 includes 62 residues, we chose to simulate only a portion of the protein centered around one of the phosphorylated threonines, T37.
The full structure includes two known phosphothreonine sites, T37 and T46.
The fragment used in the simulations includes two residues before T37 and six after, comprising a nonamer with the sequence CTTPGTLF.
It extends to both ends of the loop in which T37 resides, as seen in \figref{2MX4}.

\addfigure{2MX4}{1.0}{2MX4 in PyMOL. The fragment containing T37 used in the simulations is highlighted in green sticks.}

\subsection{Capping the termini}
Because the fragment does not have the same termini as a native peptide, we capped the ends to mitigate unrealistic interactions.
For all versions of the fragment simulated, the N-terminus was acetylated and the C-terminus was methylamidated.
These caps were added with \texttt{psfgen} using the ACE and CT2 modifications on the N- and C-terminus, respectively.

\subsection{Dephosphorylating the fragment}
The dephosphorylated version of the fragment was created by removing the phosphate ion from the original, phosphorylated fragment.
\texttt{VMD}'s autopsf tool automatically added a hydrogen to the hydroxyl group of the unphosphorylated T37.

\subsection{Creating the extended versions of the fragment}
We used PyRosetta to generate an extended fragment with the same sequence as the NMR structure, CTTPGTLF (see \figref{2MX4_p1_extended}).
To create an extended fragment with a phosphorylated threonine, we used the patch functionality of \texttt{psfgen}.
Because the bond lengths of the phosphate ion in the patched phosphothreonine were not generated correctly,
we ran 1000 steps of minimization before all simulations involving this extended fragment so that the oxygen atoms could adjust to their natural positions.

\addfigure{2MX4_p1_extended}{1.0}{The phosphorylated, extended fragment in PyMOL. Note the improper bond lengths in the phosphate ion that must be related with minimization.}

\subsection{Generating the PSF files}
We generated PSF files using either the \texttt{psfgen} command-line tool or \texttt{VMD}'s autopsf, depending on our needs.
Phosphothreonine requires additional topology and parameter files in addition to the standard protein ones.
The complete list of topology files given to \texttt{psfgen} is given below in the order we used when generating PSF files:
\begin{itemize} \itemsep 1pt
  \item top\_all36\_prot.rtf
  \item top\_all36\_na.rtf
  \item toppar\_all36\_prot\_fluoro\_alkanes.str
  \item toppar\_all36\_prot\_na\_combined.str
\end{itemize}
Because we could not get \texttt{VMD}'s autopsf tool to properly patch threonine into phosphothreonine, we used \texttt{psfgen} for the phosphorylated fragments.
The files and scripts used for PSF file generation can be found in the \texttt{psfgen} directory of the repository.

\subsection{Solvating the fragments}
All fragments were solvated using \texttt{VMD}'s solvation box tool before they were simulated.
We tried solvation boxes of size 8\AA, 10\AA, and 12{\AA} since we did not know exactly how much solvent was required to avoid wrap-around effects.
The 12{\AA} boxes proved unnecessarily large and are not used in the analysis.
The fragments were rotated to minimize volume.

\subsection{Free energy perturbation}
FEP was performed by starting with an unphosphorylated T37 and alchemically adding the phosphate group.
We tested two different versions: one in which an atom on each end of the peptide was fixed and one without any atoms fixed.
The fixed version was designed to prevent any accidental folding before alchemical transformation completed.

\subsection{File naming conventions}
Because we generated so many different versions of the fragment, we adopted consistent naming conventions for the files.
The first part of each file is \texttt{2mx4},
the second is which sites are phosphorylated (\texttt{p1} means that T37 is phosphorylated, \texttt{up1} means it is not),
the third is an optional \texttt{extended} keyword indicating that the fragment is extended,
and for solvated files there is a fourth part that indicates the size of the solvation box (\texttt{s8} means an 8{\AA} solvation box).
For the free energy perturbation simulations, there is a fifth part for alchemical files, \texttt{alch}, and for fixed files, \texttt{fixed}.

\subsection{TCL parameters}
The TCL parameters for the \texttt{NAMD} simulations remain largely unchanged from those introduced in class.
Each step was a femtosecond and energies were printed every 100 steps.
Simulations were minimized for 1000 steps.
Each simulation ran for a different amount of time depending on cluster availability and time constraints.

Different TCL parameters were used for the free energy perturbation simulations.
The additional parameters remain largely unchanged from those used in Lab 2;
a Langevin thermostat was added, there were 5000 steps of equilibration, and each of the ten FEP iterations ran for 2 nanoseconds.


\subsection{Automating simulation submission} % (fold)
\label{sub:automating_simulation_submission}

We developed a python script that takes input parameters such as the pdb file, template tickle script, time limit for the job, number of cores to use, any suffix to add to the job name, and advanced reservation ID, if available.
If ARID is given, the script fetches the resources from the ARID and uses those the job.
The script creates a separate directory for the simulation, copies the required pdb files, and generates a tickle file from the template file, filling in the parameters (input, output file names, pdb cell vector sizes, etc).
The script also creates a job bash script to submit to anthill and outputs its path, enabling us to submit jobs with different parameters with a online command.

Running a simulation with multiple cores gives significant speed improvements.
On anthill, however, jobs that require 16 cores take long time to get scheduled if there are other single-core jobs in queue.
After our 16-core jobs were still in `pending' status after 1 day, we decided to use the advanced reservation on anthill\cite{anthill-ar}.
Reserving and running jobs correctly with advanced reservation is not straight forward.
We cannot reserve nodes with `ironfs' access in advance, because `ironfs' is dynamic load resource, which tests for ironfs at the time of scheduling jobs, so the advance reservation system cannot know ahead of time.
When submitting jobs with an advanced reservation ID (ARID), the job's resource request should match exactly with that of the ARID, otherwise the job is not accepted by the scheduler.
Without the `-l ironfs' request, the simulation runs twice as slow.
As a workaround we used `-l hostname=gridiron*' request in ARID, as those nodes most of the time have access to ironfs\cite{tim:email}.

% subsection automating_simulation_submission (end)

% section methods (end)
