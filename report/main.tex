%
% Top level latex file.
%
% Author: David Kotz
% Modified by: Shrirang Mare
%
%
% Submission instruction: Uncomment line # 9 '\submittrue'


\newif\ifsubmit
\submitfalse
%\submittrue           % uncomment this line for submission. LMAKE-SUBMIT


\documentclass[twocolumn]{article}     % CHANGEME. Use conference specific style file

\usepackage[nolist]{acronym}  % handy package to manage acroymns.
\usepackage{amsmath}
\usepackage{color}
\usepackage{cite}
\usepackage{microtype}  % keeps words from spilling outside column
\usepackage{graphicx}
\usepackage{siunitx}    % useful for writing numbers and units.
\usepackage{times}
\usepackage{url}
\usepackage{xspace} 		% removes excess spaces after macros


\usepackage[pdftex, bookmarks=true]{hyperref} % creates bookmarks and hyperlinks within the pdf
\hypersetup{              % configure hyperref
  pdfcreator={git-commit: LMAKE-GIT-COMMIT},
	bookmarksnumbered=true,
  colorlinks=true,        % false: boxed links; true: colored links
  citecolor=black,        % color of links to bibliography
  filecolor=black,        % color of file links
  linkcolor=black,        % color of internal links
  urlcolor=black,         % blue
  bookmarksopen=true,
  pdfpagemode=UseNone,   % no bookmarks shown when file opened
	breaklinks=true,
	pdfstartview={FitH},
}

\pdfcompresslevel=9       % highly compressed pdf
\pagenumbering{arabic}    % Add page numbers


% Footnotes without footnote numbers
\newcommand{\footnotenonumber}[1]{{\def\thempfn{}\footnotetext{#1}}}

\newcommand{\figref}[1]{Figure~\ref{f:#1}}
% insert a narrow (1-column) figure, using the filename for the ref key.
% size is a fraction of linewidth, typically 0.9 or 1.0
% \addfigure{filename without ext}{size}{caption}
\newcommand{\addfigure}[3]{
\begin{figure}[tbp] %[tbp]
\centerline{\resizebox{#2\linewidth}{!}{\includegraphics{figs/#1}}}
\caption{\label{f:#1}#3}
\end{figure}
}
% insert a wide (2-column) figure, using the filename for the ref key.
% size is a fraction of textwidth, typically 0.9 or 1.0
% \addwidefigure{filename without ext}{size}{caption}
\newcommand{\addwidefigure}[3]{
\begin{figure*}[tbp] %[tbp]
\centerline{\resizebox{#2\textwidth}{!}{\includegraphics{figs/#1}}}
\caption{\label{f:#1}#3}
\end{figure*}
}

\ifsubmit
  \newcommand{\note}[1]{\relax}
  \newcommand{\comment}[2]{\relax}
  \newcommand{\todo}[1]{\relax}
\else
  \definecolor{commentcolor}{RGB}{0,128,255}
  \newcommand{\note}[1]{\textcolor{commentcolor}{(#1)}}
  \newcommand{\comment}[2]{\textcolor{commentcolor}{(#2 \emph{--#1})}}
  \newcommand{\todo}[1]{\textcolor{commentcolor}{(TODO: #1)}}
\fi


%%%%%%%%%%%%%%%%%%%%

\begin{document}

\title{Sample title}  % CHANGEME

\author{Author1 and Author2\\  % CHANGEME
}


\date{}

\maketitle

\ifsubmit
\else
[\textbf{NOTE:} DRAFT]

[\textbf{NOTE:} git revision: LMAKE-GIT-COMMIT]

[\textbf{NOTE:} LMAKE-GIT-STATUS]

\fi

% !TEX root = main.tex
\begin{abstract}
  	The structure of inherently disordered proteins (IDPs) is notoriously difficult to solve by either X-ray crystallography or NMR spectroscopy, but recent findings by Bah \textit{et al.} suggest that phosphorylation of certain IDPs imposes a fixed secondary structure on the proteins. We attempt to replicate these findings computationally and explain the mechanism behind the transitions to ordered conformations catalyzed by phosphorylation. Phosphorylated and unphosphorylated protein fragments in multiple starting conformations are simulated with molecular dynamics. Free energy perturbation simulations in which a phosphate ion is alchemically added to the phosphorylation site are used to further understand this phenomenon.
\end{abstract}


\ifsubmit
\else

\section*{Latex template usage info}

A comment is written using the comment macro: $\backslash$comment\{shri\}\{This is a comment\}. It is displayed as:

\comment{shri}{This is a comment}


ToDos are written as $\backslash$todo\{Write abstract.\}. They appear as :

\todo{Write abstract.}

Comments and TODO are shown when paper is built in draft mode (which is the default mode). In submit mode they are hidden. To build paper in submit mode, set submit variable to true (uncomment line 13) in main.tex or if you are using lmake.py to build run it as \texttt{./lmake.py --submit}.
\fi

% !TEX root = main.tex
\section{Introduction} % (fold)
\label{sec:introduction}



% section introduction (end)

% !TEX root = main.tex
\section{Related work} % (fold)
\label{sec:related_work}

% section related_work (end)

% !TEX root = main.tex
\section{Methods} % (fold)
\label{sec:methods}

\subsection{Producing the fragment PDB file}
The NMR structure of 4EBP2 used in this project is the same as in Bah \textit{et al.}
and has PDB ID \href{http://www.rcsb.org/pdb/explore/explore.do?structureId=2MX4}{\textcolor{blue}{2MX4}}.
Because 2MX4 includes 62 residues, we chose to simulate only a portion of the protein centered around one of the phosphorylated threonines, T37.
The full structure includes two known phosphothreonine sites, T37 and T46.
The fragment used in the simulations includes two residues before T37 and six after, comprising a nonamer with the sequence CTTPGTLF.
It extends to both ends of the loop in which T37 resides, as seen in \figref{2MX4}.

\addfigure{2MX4}{1.0}{2MX4 in PyMOL. The fragment containing T37 used in the simulations is highlighted in green sticks.}

\subsection{Capping the terminii}
Because the fragment does not have the same terminii as a native peptide, we capped the ends to mitigate unrealistic interactions.
For all versions of the fragment simulated, the N-terminus was acetylated and the C-terminus was methylamidated.
These caps were added with \texttt{psfgen} using the ACE and CT2 modifications on the N- and C-terminus, respectively.

\subsection{Dephosphorylating the fragment}
The dephosphorylated version of the fragment was created by removing the phosphate ion from the original, phosphorylated fragment.
\texttt{VMD}'s autopsf tool automatically added a hydrogen to the hydroxyl group of the unphosphorylated T37.

\subsection{Creating the extended versions of the fragment}
We used PyRosetta to generate an extended fragment with the same sequence as the NMR structure, CTTPGTLF (see \figref{2MX4_p1_extended}).
To create an extended fragment with a phosphorylated threonine, we used the patch functionality of \texttt{psfgen}.
Because the bond lengths of the phosphate ion in the patched phosphothreonine were not generated correctly,
we ran 1000 steps of minimization before all simulations involving this extended fragment so that the oxygen atoms could adjust to their natural positions.

\addfigure{2MX4_p1_extended}{1.0}{The phosphorylated, extended fragment in PyMOL. Note the improper bond lengths in the phosphate ion that must be related with minimization.}

\subsection{Generating the PSF files}
We generated PSF files using either the \texttt{psfgen} command-line tool or \texttt{VMD}'s autopsf, depending on our needs.
Phosphothreonine requires additional topology and parameter files in addition to the standard protein ones.
The complete list of topology files given to \texttt{psfgen} is given below in the order we used when generating PSF files:
\begin{itemize} \itemsep 1pt
  \item top\_all36\_prot.rtf
  \item top\_all36\_na.rtf
  \item toppar\_all36\_prot\_fluoro\_alkanes.str
  \item toppar\_all36\_prot\_na\_combined.str
\end{itemize}
Because we could not get \texttt{VMD}'s autopsf tool to properly patch threonine into phosphothreonine, we used \texttt{psfgen} for the phosphorylated fragments.
The files and scripts used for PSF file generation can be found in the \texttt{psfgen} directory of the repository.

\subsection{Solvating the fragments}
All fragments were solvated using \texttt{VMD}'s solvation box tool before they were simulated.
We tried solvation boxes of size 8\AA, 10\AA, and 12{\AA} since we did not know exactly how much solvent was required to avoid wrap-around effects.
The fragments were rotated to minimize volume.

\subsection{Free energy perturbation}
FEP was performed by starting with an unphosphorylated T37 and alchemically adding the phosphate group.
We tested two different versions: one in which an atom on each end of the peptide was fixed and one without any atoms fixed.
The fixed version was designed to prevent any accidental folding before alchemical transformation completed.

\subsection{File naming conventions}
Because we generated so many different versions of the fragment, we adopted consistent naming conventions for the files.
The first part of each file is \texttt{2mx4},
the second is which sites are phosphorylated (\texttt{p1} means that T37 is phosphorylated, \texttt{up1} means it is not),
the third is an optional \texttt{extended} keyword indicating that the fragment is extended,
and for solvated files there is a fourth part that indicates the size of the solvation box (\texttt{s8} means an 8{\AA} solvation box).
For the free energy perturbation simulations, there is a fifth part for alchemical files, \texttt{alch}, and for fixed files, \texttt{fixed}.

\subsection{TCL parameters}
The TCL parameters for the \texttt{NAMD} simulations remain largely unchanged from those introduced in class.
Each step was a femtosecond and energies were printed every 100 steps.
Simulations were minimized for 1000 steps.
Each simulation ran for a different amount of time depending on cluster availability and time constraints.

Different TCL parameters were used for the free energy perturbation simulations.
The additional parameters remain largely unchanged from those used in Lab 2;
a Langevin thermostat was added, there were 5000 steps of equilibration, and each of the ten FEP iterations ran for 2 nanoseconds.


\subsection{Automating simulation submission} % (fold)
\label{sub:automating_simulation_submission}

We wrote a python script that takes input parameters such as the input pdb file, template tickle script, time limit for the job, number of cores to use, any suffix to add to the job name, and advanced reservation ID, if available; if you give theARID
The script creates a separate directory for the simulation, copies the required pdb files, and generates a tickle file from the template file, filling in the parameters (input, output file names, pdb cell vector sizes, etc).
The script also creates a job bash script to submit to anthill and outputs its path, enabling us to submit jobs with different parameters with a online command.

VMD supports multi-core when invoked with charm.
Running a simulation on multi-core gives significant speed improvements.
On anthill, however, jobs that require 16 cores take long time to get scheduled if there are other single-core jobs in queue.
After our 16-core jobs were still in `pending' status after 1 day, we decided to use the advanced reservation on anthill\cite{anthill-ar}.
Reserving and running jobs correctly with advanced reservation is not straight forward.
We cannot reserve nodes with `ironfs' access in advance, because `ironfs' is dynamic load resource, which tests for ironfs at the time of scheduling jobs, so the advance reservation system cannot know ahead of time.
When submitting jobs with an advanced reservation ID (ARID), the job's resource request should match exactly with that of the ARID, otherwise the job is not accepted by the scheduler.
Without the `-l ironfs' request, the simulation runs twice as slow.
As a workaround we used `-l hostname=gridiron*' request in ARID, as those nodes most of the time have access to ironfs\cite{tim:email}.

% subsection automating_simulation_submission (end)

% section methods (end)

% !TEX root = main.tex
\section{Results} % (fold)
\label{sec:results}

% Get stats (bond distances and angles) from the 4 simulations.

In order to evaluate the simulations, we made several measurements for each simulation and compared these measurements across the different versions.
\figref{unphosphorylated_bond3} shows the distance between termini for the unphosphorylated fragments starting with extended and native conformations.
The unphosphorylated extended structure folds during the simulation but does not conform to a stable state,
the distance between its termini varying significantly over time.
In contrast, \figref{phosphorylated_bond3} shows that neither the extended nor native peptide changed conformation signifantly;
the distance between the termini remained constant throughout both simulations.
This could be due to the stabilizing effect of phosphorylation or simply because the simulation was not run long enough.
The simulations that produced these figures used a 10{\AA} solvation box.

\figref{unphosphorylated_extended_states} shows the conformations of unphosphorylated extended structure at different stages of simulation.
The structure samples a variety of conformations ranging from largely extended to compact, with T37 and T41 briefly coming close and then separating again.

To determine whether or not the phosphorylation of T37 affected the secondary structure of the fragment,
we calculated the $\phi/\psi$ angles of multiple residues in the peptide, as shown in \figref{ramachandran_native_s8},
which is comprised of all $\phi/\psi$ in all simulations, divided into phosphorylated and unphosphorylated simulations.
The phosphorylated peptide sampled a narrower range of $\phi/\psi$ angles than its unphosphorylated counterpart,
suggesting that the structure imposed by the phosphorylation restricts the favorable conformations of the fragment.

The structure-imposing effects can be further seen in the distance between T37 and T41;
\figref{distances} plots the distance over time between
\begin{enumerate} \itemsep 1pt
  \item the phosphate H atom of T37 and hydroxyl H atom of T41
  \item the C$\alpha$ atom of T37 and the C$\alpha$ atom of T41
\end{enumerate}
In the phosphorylated fragment, the distance between the H bonds and C$\alpha$ bonds remains nearly constant throughout the simulation.
In contrast, in the unphosphorylated fragment the distances vary considerably.
Some of the fluctuation in the H bond distances is from thermal fluctuations but the C$\alpha$ variation confirms that there is significant variation.
It is thus likely that the phosphate ion of T37 bonds with the hydroxyl group of T41, helping to maintain a constant distance between the two residues.

% Data for this figure came from:
% 2mx4_up1_extended_s10_run1
% 2mx4_up1_s10_run1_spm8
\addwidefigure{unphosphorylated_bond3}{1}{Distance between termini for unphosphorylated fragments starting with extended and native conformations.}

% Data for this figure came from:
% 2mx4_p1_extended_s10_run1_c1
% 2mx4_p1_s10_run1_c1
\addwidefigure{phosphorylated_bond3}{1}{Distance between termini for phosphorylated fragments starting with extended and native conformations.}

% Data for this figure came from:
% all simulations
\addwidefigure{ramachandran_native_s8}{1}{Each dot is a $\phi/\psi$ angle of residue taken at a different time point in some simulation.}

% Data for this figure came from:
% 2mx4_up1_extended_s10_run1
\addwidefigure{unphosphorylated_extended_states}{1}{Conformations of unphosphorylated extended structure at different stages of simulation.}

% Data for this figure came from:
% 2mx4_up1_s8
% 2mx4_p1_s8
\addwidefigure{distances}{1}{The distances over time between T37 and T41 in phosphorylated and unphosphorylated simulations with an 8{\AA} solvation box.}

% Figures: bond distances vs time for all simulations

% Figures: angles vs time for all simulations

% section results (end)

% !TEX root = main.tex
\section{Discussion} % (fold)
\label{sec:discussion}

What we did not do. Or what we could have done more.

% section discussion (end)

% !TEX root = main.tex
\section{Conclusion} % (fold)
\label{sec:conclusion}

% section conclusion (end)


\bibliographystyle{abbrvDOI.bst}
\bibliography{refs}

\end{document}
