% !TEX root = main.tex
\section{Discussion} % (fold)
\label{sec:discussion}

What we did not do. Or what we could have done more.

While there are still some reasonable doubts about our phosphorylated simulations due to their extremely low level of movement, our results are consistent with the hypothesis that it is possible to model phosphorylation based structure switching using MD.
 The high degree of movement seen in both the extended and native unphosphorylated simulations demonstrates that the unphosphorylated peptide is unable to form a stable structure on its own.
 The extremely low levels of movement seen in the native phosphorylated peptide are consistent with the phosphorylation's stabilization of the native structure.
 Unfortunately the extended phosphorylated simulation was unsuccesful making our results less conclusive than we would have liked.
 Ideally after resolving the issue of the static simulations, we could complete a larger number of the simulations shown here to capture between simulation variability.
 we could then produce a much clearer picture of both the modelability of phosphorylation based structure switching as well as the number and length of simulations necessary to detect such effects.
 We also mined the Database of Disordered Protein Prediction to produce a few frequently occuring and one highly unlikely disordered threonine phosphorylation motif.
Due to the shared proline residue between the most common sequences and the 4ebp2 sequence it is reasonable to suggest that this may be a relatively common function of threonine phosphorylation in disordered regions.
We would therefore like to further investigate the frequency of phosphorylation dependent structure switching by simulating these sequences.
 There is a great deal of further investigation possible, however our initial work suggests that MD based simulation is a promising means of approaching questions of phosphorylation based structure switching. 


% section discussion (end)
